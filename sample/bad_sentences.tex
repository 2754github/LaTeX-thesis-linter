@textlint-ja/no-synonyms:
「サーバ」と「サーバー」のような「表記揺れ」を指摘します. 

@textlint-ja/textlint-rule-no-dropping-i:
「開発してます」のような「い抜き言葉」を指摘します. 

@textlint-ja/textlint-rule-no-insert-dropping-sa:
「刺し身を食べたさそう」のような「さ入れ言葉」を指摘します. 
「これは良そうに見える」のような「さ抜き言葉」を指摘します. 

@textlint-ja/textlint-rule-no-insert-re:
「お酒は飲めれない」のような「れ足す言葉」を指摘します. 

ja-hiragana-fukushi:
「生憎 => あいにく」のような「ひらがな表記の方が読みやすい副詞」を指摘します. 
ja-hiragana-hojodoushi:
「して下さい => してください」のような「ひらがな表記の方が読みやすい補助動詞」を指摘します. 

ja-hiragana-keishikimeishi:
「する事 => すること」のような「ひらがな表記の方が読みやすい形式名詞」を指摘します. 

ja-no-abusage:
「とりつく暇もない」のような「よくある誤用」を指摘します. 

ja-no-redundant-expression:
「この文章は冗長であると言えます」のような「冗長表現」を指摘します. 

ja-no-successive-word:
「これは問題あるある文章です」のような「"間違った"同一単語の連続」を指摘します. 
「すもももももももものうち」のような文章は指摘されません. 

ja-no-weak-phrase:
「この証明は正しいかもしれない」のような「弱い表現」を指摘します. 

ja-unnatural-alphabet:
「今週末のリイr−スには対応でkない」のような「不自然なアルファベット」を指摘します. 

max-comma:
「1文章中における, 指定した数を超えるカンマの出現」を指摘します. 
「この, 文章, は, カンマ, が, 多すぎ, る. 」など. 

max-kanji-continuous-len:
「指定した数を超える漢字の連続」を指摘します. 
「日本応用数理学会」など. 

no-double-negative-ja:
「この文章は問題なくもない」のような「二重否定」を指摘します. 

no-doubled-conjunction:
「早起きした. しかし, 定刻には間に合わなかった. しかし, 無事, 学会会場に到着した. 」
のような「同一接続詞の連続」を指摘します. 

no-doubled-conjunctive-particle-ga:
「早起きしたが, 定刻には間に合わなかったが, 無事, 学会会場に到着した. 」
のような「逆接の接続助詞"が"の連続」を指摘します. 

no-doubled-joshi:
「私は彼は好きだ」のような「同一助詞の連続」を指摘します. 

no-dropping-the-ra:
「この距離からでも見れる」のような「ら抜き言葉」を指摘します. 

no-hankaku-kana:
「この文章はダメダメです」のような「半角カタカナ」を指摘します. 

no-mix-dearu-desumasu:
「吾輩は猫である. 名前はまだないです. 」のような「常体と敬体の混在」を指摘します. 

no-mixed-zenkaku-and-hankaku-alphabet:
「ABC」のような「アルファベットの全角と半角の混在」を指摘します. 
デフォルトで全角アルファベットの使用を禁止しています. 

no-nfd:
「ど゛う゛し゛て゛!゛」のような「問題となる濁点」を指摘します. 

prefer-tari-tari:
「気温が上がったり, 下がる. 」のような「〜たり〜たりする」を指摘します. 

spellcheck-tech-word:
「Java Script => JavaScript」のような「技術用語の誤表記」を指摘します. 

prh.yml: 表記の統一ルールを設定してあるので見てみて欲しい. 

my-rules:
全角のスペース/読点/句点/カンマ/ピリオド/丸かっこを使用しない
 = 半角のスペース/カンマ/ピリオド/丸かっこを使用する:
エラーとなる例::  、。,.()

「,」「.」の前に「␣」を挿入しない:
エラーとなる例:: , . 

「,」「.」の後に「␣」を挿入する:
エラーとなる例:: ,.

「,␣(改行)」「,(改行)」で文が終了している場合に警告する:
警告される例: , 
,

文頭が数式の場合に警告する
警告される例:
$X$をバナッハ空間とする

ファイルの最後には改行を挿入する
エラーとなる例:
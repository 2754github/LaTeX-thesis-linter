@textlint-ja/no-synonyms: サーバとサーバーの表記揺れがある. 
@textlint-ja/textlint-rule-no-dropping-i: 開発してます. 
@textlint-ja/textlint-rule-no-insert-dropping-sa: 刺し身を食べたさそう. これは良そうに見える. 
@textlint-ja/textlint-rule-no-insert-re: お酒は飲めれない. 
ja-hiragana-fukushi: 「生憎」より「あいにく」の方が読みやすい. 
ja-hiragana-hojodoushi: 「して下さい」より「してください」の方が読みやすい. 
ja-hiragana-keishikimeishi: 「する事」より「すること」の方が読みやすい. 
ja-no-abusage: 例外を補足する. 
ja-no-redundant-expression: この文章は冗長であると言えます. 
ja-no-successive-word: これは問題あるある文章です. 
ja-no-weak-phrase: この証明は正しいかもしれない. 
ja-unnatural-alphabet: 今週末のリイr−スには対応でkない. 
max-comma: この, 文章, は, カンマ, が, 多すぎ, る. 
max-kanji-continuous-len: 日本応用数理学会. 
no-double-negative-ja: この文章は問題なくもない. 
no-doubled-conjunction: 早起きした. しかし, 定刻には間に合わなかった. しかし, 無事, 学会会場に到着した. 
no-doubled-conjunctive-particle-ga: 早起きしたが, 定刻には間に合わなかったが, 無事, 学会会場に到着した. 
no-doubled-joshi: 私は彼は好きだ. 
no-dropping-the-ra: この距離からでも見れる. 
no-hankaku-kana: この文章はダメダメです. 
no-mix-dearu-desumasu: 吾輩は猫である. 名前はまだないです. 
no-mixed-zenkaku-and-hankaku-alphabet: ABCのような全角アルファベットは許しません. 
no-nfd: ど゛う゛し゛て゛!゛. 
prefer-tari-tari: 気温が上がったり, 下がる. 
spellcheck-tech-word: Googleの正しい表記はgoogleではなくGoogleです. 

my-rules
句読点「、。」の使用は禁止です. 
カンマや ,ピリオドの前にはスペースが不要ですが,後にはスペースが必要です.
カンマで終わる文章は好ましくないです, 
口語のため, 使用しません. 
「よぶ」「呼ぶ」=>「いう」
 ←ここに全角スペースがあります
$X$をバナッハ空間
日本応用数理学会(JSIAM)
日本応用数理学会(JSIAM)
